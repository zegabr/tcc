\section{Visão Geral da Implementação}
\subsection{Otimização e Simplificação com uso de \emph{awk}}
Primeiramente, para que consiguíssemos fazer o \emph{CSDiff} detectar mudanças de indentação, era necessário
que o programa iterasse linha a linha, no Passo 1 da Figura~\ref{csdiff_process}.
Para tal, foi escolhido a ferramenta \emph{awk}~\cite{awk}, utilizada primeiramente para
simplificar alguns dos scripts feitos com a ferramenta \emph{sed} que foram utilizados em~\cite{clem21,heitor21}.
Esta simplificação
utilizou-se do conceito chamado de \emph{charclass} (add footnote: https://www.regular-expressions.info/charclass.html), que
permite criar um comando de substituição mais simples do que o utilizado anteriormente, que consistia em uma sequencia de
substituições \emph{sed} individuais. Essa simplificação foi posteriormente testada pelo autor utilizando
o \emph{miningframework} (utilizado nessa pesquisa, bem como nas anteriores já mencionadas), para comprovar sua equivalência em
relação ao script inicial (sequência de comandos \emph{sed}). Além disso, apesar de não ter sido
metodologicamente analisada (por não fazer parte do escopo desta pesquisa),
notou-se uma melhora na velocidade de execução do
\emph{CSDiff} de aproximadamente 3x, indicando uma diferença considerável de performance entre as ferramentas
\emph{awk} e \emph{sed}.

\subsection{Inserindo Marcadores ao Detectar Mudanças de Indentação}
Após a simplicicação acima, foi criada uma nova versão do \emph{CSDiff} cujo processo é descrito na
Figura~\ref{csdiff_process_indentation} e seu resultado ilustrada nas Figuras \ref{base_marcadores_indentacao},
\ref{left_marcadores_indentacao},
\ref{right_marcadores_indentacao}, enquanto que o resultado da execução do \emph{diff3} pode ser visto na figura
\ref{diff3_marcadores_indentacao}

\begin{figure}[ht]
	\begin{center}
		\begin{compactenum}[(1)]
			\item Iterando linha a linha nos arquivos \emph{base}, \emph{left} e \emph{right}:
			\begin{compactenum}
				\item Transforma os arquivos, fazendo com que os separadores sintáticos
				dados como entrada fiquem em linhas separadas; essas novas linhas são marcadas com a sequencia de caracteres:
				'\verb|$$$$$$$$|', e adiciona uma linha de marcadores a mais acima da linha atual
				sempre que a linha anterior está em um nível de
				indentação diferente da linha atual.
			\end{compactenum}
			\item Chama o \emph{merge} textual do Diff3, passando como entrada os arquivos gerados por 1;
			\item No arquivo resultante do Passo 2, remove as linhas extras e marcadores adicionados no Passo 1.
		\end{compactenum}
	\end{center}
	\caption{Processo do \emph{CSDiff}}\label{csdiff_process_indentation}
\end{figure}


\begin{figure}[ht]
	\begin{center}
		\lstinputlisting[language=Python]{./example/indentation/base.py_temp.py}
		\caption{Arquivo \emph{base} após a o Passo 1 da Figura
			\ref{csdiff_process_indentation}}\label{base_marcadores_indentacao}
	\end{center}
\end{figure}

\begin{figure}[ht]
	\begin{center}
		\lstinputlisting[language=Python]{./example/indentation/left.py_temp.py}
		\caption{Arquivo \emph{left} após a o Passo 1 da Figura
			\ref{csdiff_process_indentation}}\label{left_marcadores_indentacao}
	\end{center}
\end{figure}

\begin{figure}[ht]
	\begin{center}
		\lstinputlisting[language=Python]{./example/indentation/right.py_temp.py}
		\caption{Arquivo \emph{right} após a o Passo 1 da Figura
			\ref{csdiff_process_indentation}}\label{right_marcadores_indentacao}
	\end{center}
\end{figure}

\begin{figure}[ht]
	\begin{center}
		\lstinputlisting[language=Python]{./example/indentation/diff3_temp.py}
		\caption{Arquivo resultante após a execução do Passo 2 da Figura \ref{csdiff_process_indentation} nos arquivos
			\emph{base}, \emph{left} e \emph{right}. Note que agora o diff3 conseguiu resolver todos
			os conflitos de forma automática.
		}\label{diff3_marcadores_indentacao}
	\end{center}
\end{figure}

\begin{figure}[ht]
	\begin{center}
        \lstinputlisting[language=Python]{./example/indentation/csdiff.py}
		\caption{Arquivo resultante após a o Passo 3 da Figura
			\ref{csdiff_process_indentation}}\label{csdiff_indentacao}
	\end{center}
\end{figure}

Com essas modificações, fizemos o experimento descrito na seção seguinte.




