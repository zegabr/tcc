\section{Visão Geral da Implementação}\label{implementacao}
\subsection{Simplificação com uso de \emph{awk}}

Primeiramente, para que conseguíssemos fazer o \emph{CSDiff} detectar mudanças
de indentação, era necessário que o programa iterasse linha por linha, no Passo
1 da Figura~\ref{csdiff_process}. Então, ao pesquisar por ferramentas
alternativas ao \emph{sed} (que, ao contrário do \emph{awk}, faz a substituição
no arquivo inteiro de uma vez), encontramos e escolhemos utilizar a ferramenta
\emph{awk}~\cite{awk}, utilizada primeiramente para simplificar alguns dos
scripts feitos com a ferramenta \emph{sed} que foram utilizados
em~\cite{clem21,heitor21}. Esta simplificação foi feita através do conceito
chamado de
\emph{charclass}\footnote{https://www.regular-expressions.info/charclass.html}
que permite criar um comando de substituição mais simples do que o utilizado
anteriormente, que consistia em uma sequencia de substituições \emph{sed}
individuais. Além disso, essa modificação foi o que permitiu que o programa
iterasse linha por linha de uma maneira eficiente e programática.

Essa simplificação foi posteriormente testada pelo autor utilizando o
\emph{miningframework} (utilizado nessa pesquisa, bem como nas anteriores já
mencionadas), para comprovar sua equivalência em relação ao script inicial
(sequência de comandos \emph{sed}).

\subsection{Inserindo Marcadores ao Detectar Mudanças de Indentação}

Após a simplificação acima, foi criada uma nova versão do \emph{CSDiff} cujo
processo é descrito na Figura~\ref{csdiff_process_indentation} e seu resultado
ilustrada nas Figuras~\ref{base_marcadores_indentacao},
~\ref{left_marcadores_indentacao}, ~\ref{right_marcadores_indentacao}, enquanto
que o resultado da execução do \emph{diff3} nos arquivos modificados pode
ser visto na figura ~\ref{diff3_marcadores_indentacao}

% TODO: trocar pra listagem e suas referencias
\begin{figure}[ht]
	\begin{center}
		\begin{compactenum}[(1)]
			\item Iterando linha a linha nos arquivos \emph{base}, \emph{left} e \emph{right}:
			\begin{compactenum}
                \item Transforma os arquivos, fazendo com que os separadores
                    sintáticos dados como entrada fiquem em linhas separadas,
                    adicionando uma linha antes e uma depois de cada separador;
                    essas novas linhas são marcadas com a sequência de
                    caracteres: '\verb|$$$$$$$$|', e adiciona uma linha de
                    marcadores a mais acima da linha atual sempre que a linha
                    anterior está em um nível de indentação diferente da linha
                    atual.
			\end{compactenum}
            \item Chama o \emph{merge} textual do \emph{Diff3}, passando como
                entrada os arquivos gerados por 1;
            \item No arquivo resultante do Passo 2, remove as linhas extras e
                marcadores adicionados no Passo 1.
		\end{compactenum}
	\end{center}
	\caption{Processo do \emph{CSDiff} considerando mudanças de indentação}\label{csdiff_process_indentation}
\end{figure}

\begin{figure}[ht]
	\begin{center}
		\lstinputlisting[language=Python]{./example/indentation/base.py_temp.py}
		\caption{Arquivo \emph{base} após o Passo 1 da Figura
			\ref{csdiff_process_indentation}}\label{base_marcadores_indentacao}
	\end{center}
\end{figure}

\begin{figure}[ht]
	\begin{center}
		\lstinputlisting[language=Python]{./example/indentation/left.py_temp.py}
		\caption{Arquivo \emph{left} após o Passo 1 da Figura
			\ref{csdiff_process_indentation}}\label{left_marcadores_indentacao}
	\end{center}
\end{figure}

\begin{figure}[ht]
	\begin{center}
		\lstinputlisting[language=Python]{./example/indentation/right.py_temp.py}
		\caption{Arquivo \emph{right} após o Passo 1 da Figura
			\ref{csdiff_process_indentation}}\label{right_marcadores_indentacao}
	\end{center}
\end{figure}

\begin{figure}[ht]
	\begin{center}
		\lstinputlisting[language=Python]{./example/indentation/diff3_temp.py}
		\caption{Arquivo resultante após a execução do Passo 2 da Figura~\ref{csdiff_process_indentation} nos arquivos
			\emph{base}, \emph{left} e \emph{right}. Note que agora o \emph{Diff3} conseguiu resolver todos
			os conflitos de forma automática.
		}\label{diff3_marcadores_indentacao}
	\end{center}
\end{figure}

\begin{figure}[ht]
	\begin{center}
		\lstinputlisting[language=Python]{./example/indentation/csdiff.py}
		\caption{Arquivo resultante após o Passo 3 da Figura
			\ref{csdiff_process_indentation}}\label{csdiff_indentacao}
	\end{center}
\end{figure}

Como podemos ver nas
Figuras~\ref{base_marcadores_indentacao},~\ref{left_marcadores_indentacao}
e~\ref{right_marcadores_indentacao}, foi adicionada uma nova linha de
marcadores entre os dois \emph{return}s, de forma que todas as linhas
conflitantes deixaram de estar em posições consecutivas, pois a linha de
marcadores (mais especificamente o da linha 15) é idêntica em todos os
arquivos. Isso foi suficiente para que o Passo 2 da
Figura~\ref{csdiff_process_indentation} resolvesse todos os conflitos de forma
automática.

Por questão de simplicidade, utilizamos apenas dois separadores nos exemplos
acima ``(`` e ``)`` (abre parênteses e fecha parênteses). Entretanto, Python também possui outros separadores
possivelmente relevantes, como ``,`` (vírgula) e ``:`` (dois pontos), que
utilizamos na avaliação desse trabalho. Apesar disso, vale resslatar que na
linguagem, o separadoor ``:`` normalmente vem precedido de um ``)`` e/ou
seguido de uma mudança de indentação. Isso faz com que o \emph{CSDiff} crie
linhas marcadoras redundantes, e como veremos mais a frente, tem influência
direta na geração de falsos conflitos.

Com essas modificações, executamos o experimento descrito na seção seguinte.

