\section{CONCEITOS}
\subsection{Cenário de Merge}
\subsection{Falso Positivo Adicionado}
\subsection{Falso Negativo Adicionado}
\subsection{Resultado Errado de Merge}

\section{PERGUNTAS DE PESQUISA}
\subsection{A nova solu¸c˜ao de merge n˜ao estruturado, utilizando separadores, reduz a quantidade de conflitos reportados em compara¸c˜ao
ao merge puramente textual?}
\subsection{A nova solu¸c˜ao de merge n˜ao estruturado, utilizando separadores, reduz a quantidade de cen´arios com conflitos reportados em
compara¸c˜ao ao merge puramente textual?}
\subsection{A nova solu¸c˜ao de merge n˜ao estruturado, utilizando separadores, reduz a quantidade de falsos conflitos e cen´arios com falsos
conflitos reportados (falsos positivos) em compara¸c˜ao ao merge
puramente textual?}
\subsection{A nova solu¸c˜ao de merge n˜ao estruturado, utilizando separadores, amplia a possibilidade de comprometer a corretude do
c´odigo, por aumentar o n´umero de integra¸c˜oes de mudan¸cas que
interferem uma na outra, sem reportar conflitos (falsos negativos), al´em de aumentar cen´arios com falsos negativos?}
\subsection{PP5 sobre aumento de produtividade}
TODO: adicionar uma subsubsection pra cada possivel situacao desse role, após explicar o role

\section{METODOLOGIA}
\section{RESULTADOS}
TODO: copiar as subsections do perguntas de pesquisa aqui

\section{DISCUSSÃO}
\section{AMEAÇAS A VALIDADE}
