%! TeX root: ./main.tex
%% Template para dissertação/tese na classe UFPEthesis
%% versão 0.9.2
%% (c) 2005 Paulo G. S. Fonseca
%% www.cin.ufpe.br/~paguso/ufpethesis

%% Carrega a classe ufpethesis
%% Opções: * Idiomas
%%           pt   - português (padrão)
%%           en   - inglês
%%         * Tipo do Texto
%%           bsc  - para monografias de graduação
%%           msc  - para dissertações de mestrado (padrão)
%%           qual - exame de qualificação doutorado
%%           prop - proposta de tese doutorado
%%           phd  - para teses de doutorado
%%         * Mídia
%%           scr  - para versão eletrônica (PDF) / consulte o guia do usuario
%%         * Estilo
%%           classic - estilo original à la TAOCP
%%           modern  - estilo à la CUP (padrão)
%%           ugly    - formato da UFPE com parte pré-textual no formato ABNT
%%         * Paginação
%%           oneside - para impressão em face única
%%           twoside - para impressão em frente e verso (padrão)
\documentclass{ufpethesis}

%% Preâmbulo:
%% coloque aqui o seu preâmbulo LaTeX, i.e., declaração de pacotes,
%% (re)definições de macros, medidas, etc.

%% Identificação:

% Universidade
% e.g. \university{Universidade de Campinas}
% Na UFPE, comente a linha a seguir
\university{NOME DA UNIVERSIDADE}

% Modifique o comando \universitylogo para alterar o logo da universidade
% e.g.
% \renewcommand{\universitylogo}{\includegraphics{newlogo.pdf}}

% Endereço (cidade)
% e.g. \address{Campinas}
% Na UFPE, comente a linha a seguir
\address{CIDADE DA IES}

% Instituto ou Centro Acadêmico
% e.g. \institute{Centro de Ciências Exatas e da Natureza}
% Comente se não se aplicar
\institute{NOME DO INSTITUTO OU CENTRO ACADÊMICO}

% Departamento acadêmico
% e.g. \department{Departamento de Informática}
% Comente se não se aplicar
\department{NOME DO DEPARTAMENTO}

% Programa de pós-graduação
% e.g. \program{Pós-graduação em Ciência da Computação}
\program{NOME DO PROGRAMA}

% Área de titulação
% e.g. \majorfield{Ciência da Computação}
\majorfield{NOME DA ÁREA DE TITULAÇÃO}

% Título da dissertação/tese
% e.g. \title{Sobre a conjectura $P=NP$}
\title{TÍTULO DO TRABALHO}

% Data da defesa
% e.g. \date{19 de fevereiro de 2003}
\date{DATA DA DEFESA}

% Autor
% e.g. \author{José da Silva}
\author{NOME DO AUTOR}

% Orientador(a)
% Opção: [f] - para orientador do sexo feminino
% e.g. \adviser[f]{Profa. Dra. Maria Santos}
\adviser{NOME DO(DA) ORIENTADOR(A)}

% Orientador(a)
% Opção: [f] - para orientador do sexo feminino
% e.g. \coadviser{Prof. Dr. Pedro Pedreira}
% Comente se não se aplicar
\coadviser{NOME DO(DA) CO-ORIENTADOR(A)}

%% Inicio do documento
\begin{document}

%%
%% Parte pré-textual
%%
\frontmatter

% Folha de rosto
% Comente para ocultar
\frontpage

% Portada (apresentação)
% Comente para ocultar
\presentationpage

% Dedicatória
% Comente para ocultar
\begin{dedicatory}
DIGITE A DEDICATÓRIA AQUI
\end{dedicatory}

% Agradecimentos
% Se preferir, crie um arquivo à parte e o inclua via \include{}
\acknowledgements
DIGITE OS AGRADECIMENTOS AQUI

% Epígrafe
% Comente para ocultar
% e.g.
%  \begin{epigraph}[Tarde, 1919]{Olavo Bilac}
%  Última flor do Lácio, inculta e bela,\\
%  És, a um tempo, esplendor e sepultura;\\
%  Ouro nativo, que, na ganga impura,\\
%  A bruta mina entre os cascalhos vela.
%  \end{epigraph}
\begin{epigraph}[NOTA]{AUTOR}
DIGITE AQUI A CITAÇÃO
\end{epigraph}

% Resumo em Português
% Se preferir, crie um arquivo à parte e o inclua via \include{}
\resumo
DIGITE O RESUMO AQUI
% Palavras-chave do resumo em Português
\begin{keywords}
DIGITE AS PALAVRAS-CHAVE AQUI
\end{keywords}

% Resumo em Inglês
% Se preferir, crie um arquivo à parte e o inclua via \include{}
\abstract
% Palavras-chave do resumo em Inglês
\begin{keywords}
DIGITE AS PALAVRAS-CHAVE AQUI
\end{keywords}

% Sumário
% Comente para ocultar
\tableofcontents

% Lista de figuras
% Comente para ocultar
\listoffigures

% Lista de tabelas
% Comente para ocultar
\listoftables



%%
%% Parte textual
%%
\mainmatter

% É aconselhável criar cada capítulo em um arquivo à parte, digamos
% "capitulo1.tex", "capitulo2.tex", ... "capituloN.tex" e depois
% incluí-los com:
% \include{capitulo1}
% \include{capitulo2}
% ...
% \include{capituloN}



%%
%% Parte pós-textual
%%
\backmatter

% Apêndices
% Comente se não houver apêndices
\appendix

% É aconselhável criar cada apêndice em um arquivo à parte, digamos
% "apendice1.tex", "apendice.tex", ... "apendiceM.tex" e depois
% incluí-los com:
% \include{apendice1}
% \include{apendice2}
% ...
% \include{apendiceM}


% Bibliografia
% É aconselhável utilizar o BibTeX a partir de um arquivo, digamos "biblio.bib".
% Para ajuda na criação do arquivo .bib e utilização do BibTeX, recorra ao
% BibTeXpress em www.cin.ufpe.br/~paguso/bibtexpress
\nocite{*}
\bibliographystyle{plain}
\bibliography{biblio}

% Cólofon
% Descomente para incluir uma pequena nota com referência à UFPEThesis
%\colophon

%% Fim do documento
\end{document}
