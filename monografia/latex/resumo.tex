\resumo
A prática de desenvolvimento de software, há muito tempo deixou de ser uma tarefa
que era realizada por somente uma pessoa, pois, com o avanço da tecnologia, sistemas
cada vez mais complexos foram sendo criados fazendo com que muitas pessoas trabalhassem no
mesmo projeto. Por conta disso, ferramentas de controle de versionamento de
código foram criadas, permitindo que múltiplos desenvolvedores trabalhassem modificando
o mesmo trecho de código simultaneamente. Porém, essas modificações simultâneas
podem gerar conflitos quando feitas em um mesmo pedaço de código, o que impacta
negativamente na produtividade de um time. Ao decorrer do tempo, diversas formas
de como detectar conflitos na junção de versão de códigos foram criadas, dentre
elas: linha a linha, estruturada e semiestruturada. Neste trabalho,
é proposto uma extensão para uma ferramenta semiestruturada de detecção de conflitos
ja existente, o \emph{CSDiff} \cite{clementino2021textual}, de forma que ela
utilize indentação como separador da linguagem, permitindo assim que, durante a
detecção de conflitos em linguagens com poucos separadores sintáticos, ainda seja
possivel permitir uma redução de falsos conflitos, consequentemente melhorando a
produtividade de um time.
\begin{keywords}
	Processo de Merge, Desenvolvimento Colaborativo, Merge Textual, Merge
	Estruturado, Separadores sintáticos
\end{keywords}

\abstract
The practice of software development has long ceased to be a task performed by
only one person, as with the advancement of technology, increasingly complex
systems have been created, causing many people to work on the same project.
Therefore, code version control tools were created, allowing multiple developers
to work on modifying the same piece of code simultaneously. However, these simultaneous
modifications can generate conflicts when made on the same piece of code, negatively
impacting a team's productivity. Over time, several ways of detecting conflicts
in the merging of code versions have been created, including line-by-line, structured,
and semi-structured. In this work, an extension is proposed for an existing
semi-structured conflict detection tool, the \emph{CSDiff} \cite{clementino2021textual},
so that it uses indentation as a language separator, thus allowing, during
conflict detection in languages with few syntactic separators, a reduction in
false conflicts, consequently improving a team's productivity.
\begin{keywords}
	Merge Process, Collaborative Development, Textual Merge, Structured Merge, Syntactic Separators.
\end{keywords}

