Na academia, vários trabalhos foram feitos para aprofundar os conhecimentos em cima das abordagens e
ferramentas de merge. Entre eles, podemos citar inicialmente dois trabalhos que serviram como base para esta pesquisa.
Temos \citeauthor{clem21}, que propôs uma nova ferramenta de merge textual que tenta simular uam abordagem semiestruturada
utilizando os separadores da linguagem Java. Também temos \citeauthor{heitor21}, que extende a mesma ferramenta e analisa os
resultados dela ao ser executado em amostras de outras linguagens que ainda não tinham sido testadas (TypesCript e Ruby).

Em um estudo cujo objetivo era entender as peculiaridades do diff3, \citeauthor{khan07} formaliza, através de fórmulas e teoremas,
algumas propriedades particulares do algoritmo do diff3, que inicialmente parecem ser intuitivas e corretas, mas que no fim não
são.

Considerando a área de merge semiestruturado, temos \cite{apel11} onde a ferramenta FSTMerge é proposta e analisada. Temos
também o trabalho de \citeauthor{cavalcanti17}, onde ele
discorre sobre merge semiestruturado e seus resultados quando comparados ao uso do merge não
estruturado levando em consideração a redução de conflitos.
Em outra publicação \cite{cavalcanti19}, Cavalvanti faz uma comparação entre o merge semiestruturado e o estruturado.
Ele aponta que o primeiro tende a apresentar mais falsos positivos, enquanto o segundo apresenta mais falsos negativos.
Essa observação sugere a possibilidade de explorar um equilíbrio entre as diferentes ferramentas de abordagens
distintas, buscando um meio termo entre uma abordagem textual e as abordagens
semiestruturadas e estruturadas.

Em \citeauthor{accioly18}, o autor conduz uma pesquisa com o objetivo de examinar e classificar os
padrões de conflitos em projetos de código aberto escritos em Java,
com o intuito de identificar quais tipos de padrões na estrutura do
código estão associados a diferentes tipos de conflitos.

