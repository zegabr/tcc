\section{Merge Estruturado}
Como alternativa ao uso de merge não estruturado, existem as abordagens
semiestruturadas ou completamente estruturadas. Ao contrário da abordagem não
estruturada, essas abordagens levam em consideração a estrutura sintática da
linguagem de programação para identificar conflitos com maior precisão e resolvê-los de 
forma mais correta. Essas abordagens criam árvores sintáticas para
cada versão dos arquivos a serem integrados (\emph{base}, \emph{left} e \emph{right}) 
e comparam essas árvores para identificar nós comuns
e adições ou remoções em cada árvore. Dessa forma, cada elemento sintático
é representado em nós distintos, e conflitos são sinalizados quando as mudanças
a serem integradas estão relacionadas ao mesmo nó da árvore. Isso significa
que, em vez de usar linhas como a unidade básica para comparação, essas ferramentas usam
nós sintáticos como unidade.

Essas ferramentas estruturadas e semiestruturadas conseguem evitar falsos conflitos
encontrados na abordagem não estruturada. Por exemplo, duas situações em
que dois desenvolvedores adicionam separadamente dois novos métodos com diferentes
assinaturas em uma mesma área do texto podem ser conciliadas com sucesso.
As mudanças ocorrem na mesma linha, mas cada declaração é representada por
um nó diferente, pois o identificador do método é parte do nó,
e os dois nós são mantidos na árvore resultante da integração.

TODO: adicionar figuras do exemplo decente

Da mesma forma, uma ferramenta estruturada para Python evitaria o
conflito apresentado na Figura 4. A ferramenta, utilizando a estrutura da
linguagem, identifica que, apesar das mudanças representadas na Figura 2 e na
Figura 3 ocorrerem na mesma linha, elas estão associadas a nós diferentes na árvore
sintática. A ferramenta então concilia as mudanças em uma versão resultante que contém a
nova condição proposta por left e a extração de constante proposta por right,
evitando o falso conflito.
\section{Merge Semiestruturado e Estruturado}
\section{Merge Não Estrurado com Separadores}
