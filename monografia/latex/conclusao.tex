Propomos neste artigo uma modificação sobre uma ferramenta já existente, o \emph{CSDiff}~\cite{clem21},
visando analisar sua performance para a linguagem Python. Testamos a versão já existente, criada em~\cite{heitor21},
e a versão com a modificação proposta neste trabalho, que faz com que a ferramenta considere mudanças de indentação na
tentativa de separar escopos de programa Python. Analisamos também a influência de utilizar dois conjuntos de separadores da
linguagem. Seguimos a mesma metodologia utilizada pelos dois trabalhos, com adição de uma nova análise
de aumento de produtividade criada pelo autor.

Encontramos resultados similares aos das pesquisas anteriores em relação ao aumento na quantidade de conflitos e redução na
quantidade de cenários com conflitos ao comparar os resultados da ferramenta diff3 com os resultados daa ferramenta \emph{CSDiff}.
Considerando a adição de Falsos Positivos, vemos uma pequena desvantagem do diff3 em relação ao \emph{CSDiff}, considerando número de
cenários, mas o oposto ocorrendo para o número de arquivos. Considerando Falsos Negativos adicionados, notamos uma pequena
desvantagem do \emph{CSDiff} em relação ao \emph{Diff3} ao considerar o núemero de
arquivos, mas uma grande desvantagem ao considerar o número de cenários

Por fim, ao analisar o aumento de produtividade, como definido na seção~\ref{concept_PP5}, notamos um aumento de
produtividade maior na versão que não considera indentação, e utilizando um conjunto menor de separadores.


